
\documentclass[12pt,a4paper]{article}
\usepackage{geometry}
\geometry{left=2.5cm,right=2.5cm,top=2.0cm,bottom=2.5cm}
\usepackage[english]{babel}
\usepackage{amsmath,amsthm,amssymb}
\usepackage{amsfonts}
\usepackage[longend,ruled,linesnumbered]{algorithm2e}
\usepackage{fancyhdr}
\usepackage[fontset=windows]{ctex}
\usepackage{array}
\usepackage{listings}
\usepackage{color}
\usepackage{graphicx}
\usepackage{amssymb}
\usepackage{algorithm}
\usepackage{algorithmic}


\begin{document}


\title{
{\heiti《高等运筹学》第 {$2$} 次作业
\footnote{
 缺货率处理为线性,因此在约束条件为线性的情况下才能求解。在单周期问题中,未查到缺货率相关概念。
}
}
}
\date{}

\author{
姓名:\underline{王章任}~~~~~~
学号:\underline{2024201050128}~~~~~~}

\maketitle

\noindent
\section*{\heiti \color{red}{报童模型拓展}}
\noindent
{\bf 题目1:}假设报童模型中需求量的概率分布为$F(x)$,成本参数为$c$和$p$,订货量受到限制$(Q\leq K)$,试分析在需求量的概率分布为$F(x)$的情况下,最优订货量$Q^*$的计算方法。



\vspace{5pt}
\noindent
{\bf 解:}
\begin{align}
    \max_Q\text{\ }\pi \left( Q^{*} \right) &=\left( P-S \right) \left( \int_0^Q{xf\left( x \right) dx+\int_Q^{\infty}{Qf\left( x \right) dx}} \right) -\left( W-S \right) Q \\
    s.t.&\ Q\le K
\end{align}


由于约束条件为线性的,则判断其目标函数$\pi $为经典的报童模型的求解最大收益的函数,则可以判断其一节可导,因此可以使用拉格朗日乘子法求解其最优解,采用KKT条件求解其最优解,即:\\
拉格朗日乘子式:

\begin{equation}
    L\left( Q,\lambda  \right) =\pi \left( Q \right) +\lambda \left( K-Q \right)
\end{equation}

对其$Q$求导,得到:
\begin{equation}
    \frac{\partial L\left( Q,\lambda  \right)}{\partial Q} =\left( P-S \right) F\left( Q\right) -\left( W-S \right) -\lambda =0
\end{equation}

当$\lambda >0$时,有:
\begin{equation}
    \lambda \left( K-Q \right) =0   
\end{equation}
联立式(4)和式(5)可得:
\begin{equation}
    \left( P-S \right) F\left( K\right) -\left( W-S \right) =\lambda              
\end{equation}
对其进行验证原问题与$\lambda$是否满足条件,可以得到:原问题满足条件且$\lambda$>0。

当$\lambda$=0时,有:
\begin{equation}
    Q<K
\end{equation}
联立式(4)和式(7)可得:
\begin{equation}
    F\left(Q\right)=\frac{S-W}{S-P} >0
\end{equation}

对其进行验证原问题与$\lambda$是否满足条件,可以得到:原问题满足条件且$\lambda$=0。

综上所述,可以得到最优解$Q^*$的计算方法为:
\begin{equation}
    \left\{ \begin{matrix}
        F\left(Q\right)=\frac{S-W}{S-P}  &	 \lambda=0	\\
        Q=K  &	\lambda>0	\\
    \end{matrix} \right. 
\end{equation}



\vspace{5pt}
\noindent
{\bf 题目2:}假设报童模型中需求量的概率分布为$F(x)$,成本参数为$c$和$p$,缺货率满足$(\Pr\left(D > Q\right)\leq \alpha )$,试分析在需求量的概率分布为$F(x)$的情况下,最优订货量$Q^*$的计算方法。
\vspace{5pt}
\noindent \\
{\bf 解:}
这里只需要对其约束条件$Pr\left(D > Q\right)\leq \alpha$进行判定,只有在约束条件为线性的条件下才能求解。
\begin{align}
    \max_Q\text{\ }\pi \left( Q^{*} \right) &=\left( P-S \right) \left( \int_0^Q{xf\left( x \right) dx+\int_Q^{\infty}{Qf\left( x \right) dx}} \right) -\left( W-S \right) Q \\
    s.t.&\Pr\left(D > Q\right)\leq \alpha
\end{align}


这里对其缺货率在单周期问题中作如下假设,由于商品是否缺货会呈现一定的分布形式,其分布函数为$F\left(Q\right)$,则缺货率为$1-F\left(Q\right)$。为了满足线性条件,这里定义为$1-\frac{Q}{D} $约束条件即为:
\begin{equation}
  1-\frac{Q}{D}\leq \alpha
\end{equation} \\
拉格朗日乘子式:
\begin{equation}
    L\left( Q,\lambda  \right) =\pi \left( Q \right) +\lambda \left( 1-\frac{Q}{D}-\alpha \right)
\end{equation} \\
对其$Q$求导,得到:
\begin{equation}
    \frac{\partial L\left( Q,\lambda  \right)}{\partial Q} =\left( P-S \right) F(Q) -\left( W-S \right) -\frac{\lambda}{D} =0
\end{equation}
当$\lambda >0$时,有:
\begin{align}
    \lambda \left( 1-\frac{Q}{D}-\alpha \right) =0
\end{align}
得出
\begin{align*}
  Q &=\left( 1-\alpha \right) D \\
  \lambda &=\left( P-S \right) F\left( \left( 1-\alpha \right) D \right) -\left( W-S \right)>0
\end{align*}
当$\lambda$=0时,有:
\begin{equation}
    Q<D\left( 1-\alpha \right)
\end{equation}
则$F \left(Q\right)$得出
\begin{equation}
    F\left(Q\right)=\frac{S-W}{S-P} >0
\end{equation}
综上所述,可以得到最优解$Q^*$的计算方法为:
\begin{equation}
    \left\{ \begin{matrix}
        Q =\left( 1-\alpha \right) D &	\lambda >0	\\
        F\left(Q\right)=\frac{S-W}{S-P}  &	\lambda =0	\\
    \end{matrix} \right.
\end{equation}

\vspace{5pt}
\noindent
{\bf 题目3:}假设报童模型中需求量的概率分布为$F(x)$,成本参数为$c$和$p$,订货量受到限制$(Q\leq K)$,缺货率满足$(\Pr\left(D > Q\right)\leq \alpha )$,试分析在需求量的概率分布为$F(x)$的情况下,最优订货量$Q^*$的计算方法。
\vspace{5pt}
\noindent \\
{\bf 解:}
\begin{align}
    \max_Q\text{\ }\pi \left( Q^{*} \right) &=\left( P-S \right) \left( \int_0^Q{xf\left( x \right) dx+\int_Q^{\infty}{Qf\left( x \right) dx}} \right) -\left( W-S \right) Q \\
    s.t.&\Pr\left(D > Q\right)\leq \alpha \\
    &Q\le K
\end{align}
拉格朗日乘子式:
\begin{equation}
    L\left( Q,\lambda  \right) =\pi \left( Q \right) +\lambda_2\left(Q-K\right)+\lambda_2 \left( 1-\frac{Q}{D}-\alpha \right)
\end{equation} \\
对其$Q$求导,得到:
\begin{equation}
    \frac{\partial L\left( Q,\lambda_1,\lambda_2  \right)}{\partial Q} =\left( P-S \right) F(Q) -\left( W-S \right) +\lambda_1 -\frac{\lambda_2}{D} =0
\end{equation}
当$\lambda_1 >0,\lambda_2>0$时,有:
\begin{align*}
    &Q = K \\
    &Q =\left( 1-\alpha \right) D 
\end{align*}
且满足:
\begin{align*}
    W-S-\left( P-S \right) F\left( K \right) =\lambda_1-\frac{\lambda_2 (1-\alpha )}{K}
\end{align*}
当$\lambda_1 >0,\lambda_2=0$时,有:
\begin{align*}
    &Q = K \\
    &Q <\left( 1-\alpha \right) D 
\end{align*}
且验证:
\begin{align*}
    \lambda_1 = W-S-\left( P-S \right) F\left( K \right) <0
\end{align*}
则判断其不成立,因此不考虑$\lambda_1 >0,\lambda_2=0$的情况。
当$\lambda_1 =0,\lambda_2 >0$时,有:
\begin{align*}
    &Q < K \\
    &Q =\left( 1-\alpha \right) D 
\end{align*}
且验证:
\begin{align*}
    \lambda_2 =D \left(W-S\right)-D \left( P-S \right) F\left( Q \right) >0
\end{align*}
则判断其成立。   
当$\lambda_1 =0,\lambda_2 =0$时,有:
\begin{align*}
    &Q < K \\
    &Q < \left( 1-\alpha \right) D 
\end{align*}
且验证:
\begin{align*}
    F\left( Q \right) =\frac{S-W}{S-P} >0
\end{align*}
则判断其成立。\\
综上所述,可以得到最优解$Q^*$的计算方法为:
\begin{equation}
    \left\{ \begin{matrix}
        Q =K=\left( 1-\alpha \right) D &	\lambda_1 >0,\lambda_2>0	\\
        Q =\left( 1-\alpha \right) D &	\lambda_1 =0,\lambda_2>0	\\
        F\left( Q \right) =\frac{S-W}{S-P}  &	\lambda_1 =0,\lambda_2 =0	\\
    \end{matrix} \right.
\end{equation}

     

\vspace{10pt}
\noindent


\end{document}\
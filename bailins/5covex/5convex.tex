\documentclass[12pt,a4paper]{article}
\usepackage{geometry}
\geometry{left=2.5cm,right=2.5cm,top=2.0cm,bottom=2.5cm}
\usepackage[english]{babel}
\usepackage{amsmath,amsthm,amssymb}
\usepackage{amsfonts}
\usepackage{fancyhdr}
\usepackage[fontset=windows]{ctex}
\usepackage{array}
\usepackage{listings}
\usepackage{amssymb}
\usepackage{fontspec} % 使用 fontspec 包
\setmainfont{Times New Roman} % 设置主字体为 Times New Roman


\begin{document}


\title{
{\heiti《高等运筹学》第~{$5$} 次作业
% \footnote{
%  缺货率处理为线性,因此在约束条件为线性的情况下才能求解。在单周期问题中,未查到缺货率相关概念。
% }
}
}
\date{}

\author{
	姓名:\underline{王章任}~~~~~~
	学号:\underline{2024201050128}~~~~~~}

\maketitle

% \noindent
% % \section*{\heiti \color{red}{报童模型拓展}}
\noindent
{\bf 题1(1):}For each of the following functions determine whether it is convex, concave quasiconvex, or quasiconcave.\\
(a)~ \(f(x_1,x_2)=x_1x_2\) on \(\mathbf{R}_{++}^2\).

\vspace{5pt}
\noindent
{\bf 解:}
\(f\)的海塞矩阵为:
\begin{align*}
    \mathbf{\bigtriangledown} ^2f(x)=
    \begin{bmatrix} 
        0 & 1\\
        1 & 0 
    \end{bmatrix} 
\end{align*}
\(\bigtriangledown ^2f\) 既不是正定也不是负定,因此函数\(f\)不是凸函数也不是凹函数。它是准凹的,其超集为:
\begin{align*}
    \{(x_1,x_2) \in \mathbf{R}_{++}^2 | x_1x_2 \geq \alpha\}
\end{align*}
\\

\vspace{5pt}
\noindent
{\bf 题1(2):}~ \(f(x)=||Ax-b||,A \in \mathbb{R}^{m\times n} ,b \in \mathbb{R}^n\).Hint:composition.

\vspace{5pt}
\noindent
{\bf 解:}
2范数首先可知是凸的,函数\(f(x)=||Ax-b||\)可以看做凸函数\(g(y)=||y||\)与放射变化\(h(x)=Ax-b\)的组合,其中放射变化的\(h(x)\)为凸函数。有复合函数满足凸性的条件:
\begin{enumerate}
    \item \(h,g\) 是凸函数,\(\hat{g}\)是非减。
    \item \(h,g\) 是凸函数,\(\hat{g}\)是非增。
\end{enumerate}
可知,\(f(x)\)是凸函数。\\

\vspace{5pt}
\noindent
{\bf 题2:}Suppose \(f:\mathbb{R}^n \rightarrow \mathbb{R}\) is convex with  \(\mathbf{dom}~f:\mathbb{R}\) , and bounded above on \(\mathbb{R}\). Show that \(f\) is a constant.
\vspace{5pt}
\noindent \\
{\bf 解:}由于:\(f\)在\(\mathbf{dom}~f=\mathbb{R}^n\)为凸函数,则对于任意的\(x,y \in \mathbb{R}^n\)和\(\theta  \in [0,1]\)都有:
\begin{align*}
    f(\theta x+(1-\theta)y) \leq \theta f(x)+(1-\theta)f(y)
\end{align*}

由于:在\( \mathbb{R}^n\)中有界,则存在一个实数\(M\),使得对于所有\(x \in \mathbb{R}^n\)有:
\begin{align*}
    f(x)\le M
\end{align*}

我们做出假设:\(f\)不是常数,那么存在\(x_1,x_2 \in \mathbb{R}^n\)使得\(f(x_1)\neq f(x_2)\),不失一般性,假设\(f(x_1)\le f(x_2)\)。
给定一个凸组合\(z \in \mathbf{dom}~f\),对于任意的\(\lambda \in (0,1)\),满足:
\begin{align*}
    z=\lambda x_1+(1-\lambda)x_2
\end{align*}
根据\(f\)的凸性
\begin{align*}
    f(z)\le \lambda f(x_1)+(1-\lambda)f(x_2)
\end{align*}
当\(\lambda \rightarrow 0\)时,\(z \rightarrow x_2\)不等式变为:
\begin{align*}
    \lim_{\lambda \to 0}f(z)&\le \lim_{\lambda \to 0}(\lambda f(x_1)+(1-\lambda)f(x_2)) \\
    f(z)& \le f(x_2)
\end{align*} 
当\(\lambda \rightarrow 1\)时,\(z \rightarrow x_1\)不等式变为:
\begin{align*}
    \lim_{\lambda \to 1}f(z)&\le \lim_{\lambda \to 1}(\lambda f(x_1)+(1-\lambda)f(x_2)) \\
    f(z)& \le f(x_1)
\end{align*} 

若\(f(x_1)\neq f(x_2)\),则与有界性\(f(z)\le M\)矛盾。只有当\(f(x_1)=f(x_2)\)时上诉不等式成立。
因此,我们假设\(f\)不是常数的假设不成立。因此\(f\)必须是常数函数。

\vspace{5pt}
\noindent
{\bf 题3:}
Derive the conjugate of \(f(x)=\frac{1}{x},x > 0 \)
\vspace{5pt}
\noindent \\
{\bf 解:}
共轭函数的形式为:
\begin{align*}
    f^*(y)=\sup _{x \in \mathbf{dom} f} \{y^Tx-f(x)\}
\end{align*}
带入\(f(x)=\frac{1}{x},x > 0 \),并对\(x\)求偏导:
\begin{align*}
    &\frac{\partial (xy- \frac{1}{x})}{\partial x}=0 \\
    &\Rightarrow  y+\frac{1}{x^2}=0
\end{align*}
由此可知\(f^*(y)\)的表达式:
\begin{align*}
    f^*(y)=y\sqrt{-y^{-1}} -\frac{1}{\sqrt{-y^{-1}}}=-\sqrt{-4y}(y<0)
\end{align*}


\vspace{10pt}
\noindent

\end{document}\
\documentclass[12pt,a4paper]{article}
\usepackage{geometry}
\geometry{left=2.5cm,right=2.5cm,top=2.0cm,bottom=2.5cm}
\usepackage[english]{babel}
\usepackage{amsmath,amsthm,amssymb}
\usepackage{amsfonts}
\usepackage{fancyhdr}
\usepackage[fontset=windows]{ctex}
\usepackage{array}
\usepackage{listings}
\usepackage{amssymb}
\usepackage{fontspec} % 使用 fontspec 包
\setmainfont{Times New Roman} % 设置主字体为 Times New Roman


\begin{document}


\title{
{\heiti《高等运筹学》第 {$4$} 次作业
% \footnote{
%  缺货率处理为线性,因此在约束条件为线性的情况下才能求解。在单周期问题中,未查到缺货率相关概念。
% }
}
}
\date{}

\author{
	姓名:\underline{王章任}~~~~~~
	学号:\underline{2024201050128}~~~~~~}

\maketitle

% \noindent
% % \section*{\heiti \color{red}{报童模型拓展}}
\noindent
{\bf 题1(a):}Which of the following sets are convex?\\
(a)~ A $slab$, $i.e.$, a set of the form$\{x \in \mathbf{R}^n| \alpha \leq a^T \le \beta\}$

\vspace{5pt}
\noindent
{\bf 解:}
这是两个半空间的形成的交集,所以构造成其凸集。因此该集合是凸集。\\

\vspace{5pt}
\noindent
{\bf 题1(b):}~The set.$x+S_2 \subseteq S_1 $ ,where  $S_1,S_2 \subseteq\mathbf{R}^n $ with $S_1$ convex.

\vspace{5pt}
\noindent
{\bf 解:}
如果$x+y \subseteq S_1 $ ,对于所有的 $y \in S_2$,$x+S_2 \subseteq S_1$ 是一个凸集 .因此:
\begin{align*}
	\{x|x+S_2 \subseteq S_{1}\}=\bigcap_{y \in S_2}\{x|x+y \subseteq S_{1}\}=\bigcap_{y \in S_2} (S_{1}-y)
\end{align*}
$S_1-y$为凸集的交集也为凸集.\\

\vspace{5pt}
\noindent
{\bf 题2:}Let~$C_1,C_2 \subseteq \mathbb{R}^n $~be convex sets. Show that $S=\theta_1C_1+\theta_2C_2$,where
\begin{align*}
	C_1+C_2:=\{z\in\mathbb{R}^n|z=z_1+z_2,z_1\in C_1,z_2\in C_2\},
\end{align*}
is convex for any \(\theta_1,\theta_2 \in \mathbb{R}\).

\vspace{5pt}
\noindent \\
{\bf 解:}要证明$S=\theta_1C_1+\theta_2C_2$是凸集,我们需要证明任意两个点\(x,y \in S\)和对任意的\(\lambda \in (0,1),\)的点\(\lambda x+(1-\lambda)y \in S\)
使用\(C_1\)和\(C_2\)的元素表示\(x\)和\(y\):
\begin{align*}
	x=\theta_1 x_1+\theta_2 x_2 \quad y=\theta_1 y_1+\theta_2 y_2
\end{align*}
其中\(x_1,y_1 \in C_1 \)和\(x_2,y_2 \in C_2\)

由\(z=\lambda x +(1-\lambda)y\)带入得到:
\begin{equation*}
	z=\lambda(\theta_1 x_1+\theta_2 x_2)+(1-\lambda)(\theta_1 y_1+\theta_2 y_2) \\
	=\theta_1(\lambda x_1+(1-\lambda)y_1)+\theta_2(\lambda x_2+(1-\lambda)y_2)
\end{equation*}
由于\(C_1\)和\(C_2\)属于凸集,
\begin{align*}
	z_1=\lambda x_1+(1-\lambda)y_1 \in C_1 \quad z_2 = \lambda x_2+(1-\lambda)y_2 \in C_2
\end{align*}
所以\(z \in S\),所以\(S\)是凸集。\\

\vspace{5pt}
\noindent
{\bf 题3:}Show that if $S_1$ and $S_2$ are convex sets in $\mathbf{R}^{m \times n}$,then so is their partial sum
\begin{align*}
	S=\{(x,y_1+y_2)|x \in \mathbf{R}^m,y_1,y_2 \in \mathbf{R}^n ,(x,y_1) \in S_1,(x,y_2) \in S_2\}
\end{align*}
\vspace{5pt}
\noindent \\
{\bf 解:}
列举出两个点$(\bar{x},\bar{y_1}+\bar{y_2})$,$(\tilde{x},\tilde{y}_1+\tilde{y}_2) \in S_1,i.e.$,这个点同时满足:
\begin{align*}
	(\bar{x},\bar{y}_1)\in S_1,\quad(\bar{x},\bar{y}_2)\in S_2,\quad(\tilde{x},\tilde{y}_1)\in S_1,\quad(\tilde{x},\tilde{y}_2)\in S_2.
\end{align*}
构造出一个凸组合,令$(0\le \theta \le 1)$
\begin{align*}
	\theta(\bar{x},\bar{y}_1+\bar{y}_2)+(1-\theta)(\tilde{x},\tilde{y}_1+\tilde{y}_2)=(\theta\bar{x}+(1-\theta)\tilde{x},(\theta\bar{y}_1+(1-\theta)\tilde{y}_1)+(\theta\bar{y}_2+(1-\theta)\tilde{y}_2))
\end{align*}
由于:
\begin{align*}
	(\theta \bar{x}+(1-\theta)\tilde{x},\theta\bar{y}_1+(1-\theta)\tilde{y}_1) \in S_1, \quad (\theta \bar{x}+(1-\theta)\tilde{x},\theta\bar{y}_2+(1-\theta)\tilde{y}_2) \in S_2
\end{align*}
我们可知,$S_1,S_2$是凸集,由凸集相加得到的是凸集的定理可知,$S$是凸集。


\vspace{10pt}
\noindent

\end{document}\